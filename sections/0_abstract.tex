%Single paragraph up to 250 words. Mini-version of the paper that includes context, state of the art, why it is not good enough, the research question, the methods, the evaluation and conclusions.
Contemporary test methods for the perception subsystem of automated vehicles require a characterization of true-positive, false-positive, and false-negative object tracks. % general term for TP, FP, FN
The literature defines these concepts in various different ways, which impedes comparability across different works. 
To overcome this issue, this paper provides an exhaustive model to define these concepts as formally as possible.
The model generally describes the criteria for association of tracks under test to reference tracks. 
Emphasized details are object distance functions in state space including penalties and thresholds, multi-object association algorithms, reference data and labeling characteristics, geometric issues like fields of view and semantic areas, temporal issues like incomplete tracks, and generalizations to probabilistic world models and asynchronous time stamps. 
While a majority of aspects can be formalized, a case study illustrates the remaining challenges towards an entirely formal description. 



